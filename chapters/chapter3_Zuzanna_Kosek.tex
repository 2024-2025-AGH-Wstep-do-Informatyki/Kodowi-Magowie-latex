% \documentclass{article}
% \usepackage{graphicx} % do wstawiania zdjęć
% \usepackage{amsmath} % do wyrażeń matematycznych

% \begin{document}
% \title{ZADANIE}
% \author{Zuzanna Kosek}
% \date{\today}
% \maketitle
\section{Zuzanna Kosek}

\subsection{Wyrażenie matematyczne}
Wyrażenie matematyczne:
\[ x^n + y^n = z^n \]

\subsection{ZDJĘCIE}
\begin{figure}[h]
    \centering
    \includegraphics[width=0.25\linewidth]{pictures/a1ba4d86-c4ff-4f0f-94cb-9909135f7061.jpg}
    \label{LILU}
\end{figure}



\subsection{TABELA}
\begin{figure}[h]
    \centering
    \include{tables/table3_Zuza}
    \label{TABELA}
\end{figure}

\subsection{MOJE ZWIERZĘTA}
\begin{enumerate}
    \item LILU
    \item LUNA
    \item LUSI
\end{enumerate}

\begin{itemize}
    \item FIBI
    \item GEKKO
    \item LUCEK
\end{itemize}

\subsection{TEKST}

Bardzo lubię pieski. Mam w domu dwa \textbf{psy} które \textit{kocham}. Na zdjęciu widać \ref{LILU}, a tutaj wstawiam odniesienie do tabeli \ref{TABELA}.

Dzięki psom nasze życie staje się milion razy lepsze. Polecam każdemu mieć takiego słodziaka <3. 

% \end{document}
